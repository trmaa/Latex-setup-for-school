\documentclass[12pt]{report}

\usepackage{amsmath}
\usepackage{graphicx}
\usepackage{biblatex}

\begin{document}

\title{Moviment Circular Uniforme (MCU)}
\author{Pablo Trik Marín}
\date{\today}
\maketitle

\tableofcontents

%------------------------------------------------------------------------------------------------

\chapter{Introducció}
\section{Cinemàtica}

El MCU estudia el moviment, i no les causes del moviment, dun objecte que orbita. És a dir, la seva posició, velocitat i aceleració a cada moment del temps.

\section{Radians}

La llargada d'una circumferència és \(2\pi R\). Els radians són definits de manera en que els radians necessaris per complir tot un cercle són \(2\pi\). Per tant, el perimetre que tenen 3 radians a una circumferència de radi 4 és \(3*4 = 12\).

\begin{equation}
	2\pi rad * R = \delta x 
\end{equation}

%------------------------------------------------------------------------------------------------

\chapter{Fórmules}
\begin{align}
	x = v*t \\ \theta = \omega*t 
\end{align}
\begin{equation}
	a_c = \frac{v^2}{R} 
\end{equation}

\end{document}
